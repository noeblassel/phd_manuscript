\chapter{Introduction}
\label{chap:introduction}

Molecular dynamics (MD) is a set of computational techniques aimed at extracting properties of physical systems from carefully designed numerical simulations.
Due to its flexibility, and steady increase in the availability of computational ressources over the last seventy years, MD has become a mainstay of computational physics, and is now routinely used in a variety of scientific applications from material science and biology.
...

\section{An overview of Molecular Dynamics}

\subsection{Elements of statistical physics}
Statistical mechanics is the rigorous attempt to reconcile the microscopic point of view, according to which a given system's many degrees of freedom evolve in time according to basic physical principles, and the macroscopic point of view, according to which only a handful of variables are relevant to describe the system's state and evolution.
In this work, we present the necessary (Gibbsian) formalism to treat the molecular systems in thermal equilibrium generally studied with MD, and in particular, we restrict our scope to classical systems.
We note however that similar a formalism has been used to study a variety of disordered systems, including .~\noe{à développer avec refs}.

\paragraph{Microstates and macrostates.}
In this thesis, we consider systems of $N>0$ point particles, representing classical atomic nuclei.
The microscopic configuration, or microstate, of such a system is described by the positions and momenta of each one of these nuclei. The microstate therefore corresponds to a point in the phase space, 
\begin{equation}
    \label{eq:01:phase_space}
    (q,p)\in \cE := \cX\times \R^d,
\end{equation}
where~$\cX$ is either the torus~$\cX = (L\T)^d$ or the full space~$\cX=\R^d$, with $d=3N$ giving the number of degrees of freedom and~$L>0$ is a fixed length parameter.
For a configuration~$(q,p)\in\cE$, $q\in\cX$ is the position variable, and~$p\in \R^d$ is the associated momentum variable.
One can also consider positions~$q\in\cX$ in more general position manifolds~$\cX$, in which case the momentum~$p\in T_q^*\cX$ is a corresponding cotangent vector, and the phase space~$\cE=T^*\cX$ is defined as the cotangent bundle of the manifold~$\cX$.

To each microstate~$(q,p)\in\cE$, we associate its energy~$H(q,p)$, which is given by the expression
\begin{equation}
    \label{eq:01:hamiltonian}
    H(q,p) = V(q) + \frac12p^\top M^{-1}p,
\end{equation}
where~$V:\cX\to\R$ is a potential energy function, and the term~$\frac12p^\top M^{-1}p$ corresponds to the kinetic energy of a given atomic configuration, with
\[
M = \begin{pmatrix}
    m_1 \I_3 & 0 & \dotsm & 0 \\
    0 & m_2 \I_3 & \dotsm & 0 \\
    \vdots & \ddots & \ddots & \vdots\\
    & \dotsm & 0 & m_N \I_3 
\end{pmatrix}
\]
the diagonal matrix encoding the atomic masses in the system.

The basic postulate of statistical physics is that the macroscopic configuration of a system, otherwise known as its macrostate, is given by a probability distribution~$\pi\in\cP(\cE)$ over the underlying set of microstates, also known as a~\textit{statistical ensemble}.
Given a physical observable~$\varphi:\cE\to\R$, the macroscopic value of~$\varphi$ is defined as the~\textit{ensemble average}:
\begin{equation}
    \label{eq:01:ensemble_average}
    \langle \varphi\rangle_{\pi} = \E_\pi\left[\varphi\right]= \int_{\cE}\varphi\,\d \pi.
\end{equation}

In this statistical description, the ensemble~$\pi$ plays a crucial role in assigning to each microstate the likelihood of underlying the observed macrostate.
It is therefore a model parameter of paramount importance, for which some specific choices are overwhelmingly used in practice, and which we now present.

\paragraph{On the choice of statistical ensemble}
The statistical ensemble~$\pi$ is generally defined in terms of a handful of macroscopic constraints, meaning that the macroscopic value of a few observables determine~$\pi$.

A common recipe to define the statistical ensemble is to define~$\pi$ as the solution to the entropy-maximization problem

\begin{equation}
    \pi\in\underset{\rho\in\mathcal{F}}{\argmax}\,S(\rho),
\end{equation}
where~$S$ is a suitable entropy functional and~$\mathcal{F}\subset\cP(\cE)$

Boltzmann--Gibbs distribution
\begin{equation}
    \mu(\d q,\d p) = \frac1{\mathcal Z}\e^{-\beta H(q,p)},\qquad \mathcal Z = \int_{\cE}\e^{-\beta H(q,p)}\,\d q\,\d p
\end{equation}

\paragraph{Hamiltonian Dynamics.}
The classical equations of motion, deduced from Newton's second law, can be written compactly using the Hamiltonian~\eqref{eq:01:hamiltonian}, via the following ordinary differential equation in phase space:
\begin{equation}
    \label{eq:01:hamiltonian_dynamics}
    \frac{\d}{\d t}\,X_t = -J\nabla H(X_t),\qquad X_t = (q_t,p_t)\in \cE,
\end{equation}
where~$J$ is the antisymmetric matrix
\begin{equation}
    \label{eq:01:J}
    J=\begin{pmatrix}
        0&\I_{d}\\-\I_{d}&0
    \end{pmatrix}.
\end{equation}
In this form, the trajectories of~\eqref{eq:01:hamiltonian_dynamics} are more commonly called~\textit{Hamiltonian trajectories}, or trajectories of~\textit{Hamiltonian dynamics}.



\paragraph{Langevin dynamics.}

\paragraph{More general families of stochastic dynamics.}

\paragraph{Some orders of magnitude.}

\subsection{Dynamical models.}

\subsection{Sampling of equilibrium properties}
\begin{equation}
    \int_{\cE} \varphi(q,p)\,\mu(\d q,\d p)
\end{equation}
\begin{equation}
    \nu(\d q) = \frac1{\mathcal Z_\nu}\e^{-\beta V},\qquad \kappa(\d p) = \left(\frac{\beta}{2\pi}\right)^{\frac{d}{2}}\left(\det\,M\right)^{-\frac12}\e^{-\frac{\beta}{2}p^\top M^{-1}p}
\end{equation}

\begin{equation}
    \left\{
        \begin{aligned}
            \d q_t &= M^{-1}p_t\,\d t,\\ \d p_t &= -\nabla V(q_t)\,\d t - \gamma M^{-1}p_t\,\d t + \sqrt{\frac{2\gamma}{\beta}}\,\d W_t,
        \end{aligned}\right.
\end{equation}

\begin{equation}
    \d X_t = -\nabla V(X_t)\,\d t + \sqrt{\frac{2}{\beta}}\,\d W_t
\end{equation}

\subsection{Sampling dynamical properties}

\section{Mathematical approaches to the problem of metastability}
\subsection{Global descriptions}
\subsection{Local descriptions}
\subsection{Numerical methods}

\section{Main contributions of this thesis}


